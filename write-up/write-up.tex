\documentclass[11pt,fleqn]{article}

\usepackage{latexsym}
\usepackage{amssymb}
\usepackage{stmaryrd}
\usepackage{amsfonts}
\usepackage{amsmath}
\usepackage{url}

% Allow more line breaks in URLs
\usepackage{xurl}

% Enable links within the document
\usepackage{hyperref}
\usepackage{titlesec}

\usepackage[
  a4paper,
  top=3.8cm,
  bottom=2.5cm,
  inner=2.5cm,
  outer=2.5cm,
  headheight=15pt,
]{geometry}

\hypersetup{
  colorlinks=true,
  linkcolor=red,
  urlcolor=red,
  breaklinks=true,
}
\urlstyle{rm} % Make URL styled fonts match hyperref's hrefs

\usepackage{cleveref}

% Credit to Gabriel Devenyi for this bibliography cfg:
% github.com/gdevenyi/mcmaster.latex
\usepackage[
  style=numeric-comp,
  backend=biber,
  sorting=none,
  backref=true,
  maxnames=99,
  alldates=iso,
  seconds=true
]{biblatex} % bibliography
\addbibresource{references.bib}

% Soure code
\usepackage{listings}

\title{\vspace{-3.5cm}CAS 703 Term Project: Validated General-Purpose Calculators}
\author{Hassan Zaker \and Jason Balaci}

\date{
	McMaster University \\ \texttt{\{zakerzah, balacij\}@mcmaster.ca}\\%
	\today
}

\usepackage{multicol}
\usepackage{todonotes}

\begin{document}

\maketitle

\tableofcontents

\newpage{}

%------------------------------------------------------------------------------
% Introduction
%------------------------------------------------------------------------------
\section{Introduction}
\label{sec:introduction}

For our term project in CAS 703~\cite{Paige7032023}, we decided to build a
language for describing calculation schemes. The descriptions should be mostly
understandable to anyone who has worked with Excel~\cite{Excel} or has used any
kind of calculation software. We aim to validate the
coherence\footnote{``Coherence'' defined by an unambiguous set of constraints
and rules.} of the calculator descriptions. Additionally, through generative
techniques, we hope to decrease the barrier to entry (as much as we can) of
basic software development of calculator programs by defining a transformation
of the calculator descriptions to various programming
languages\footnote{Notably, Java programs.}.

\subsection{Objective}

We aim to:

\begin{enumerate}

  \item design a metamodel for describing calculator programs
        (\Cref{sec:modelling}),

  \item build a concrete syntax for the metamodel, and an Integrated Development
        Environment (IDE) for said concrete syntax
        (\Cref{sec:integrated-development-environment}),

  \item design a set of rules that define ``coherence'' rules of the metamodel
        and audit instances of the metamodel for coherence
        (\Cref{sec:model-validation}), and

  \item define a transformer that converts the calculator description into
        programs and corresponding documentation
        (\Cref{sec:model-management-operations}).

\end{enumerate}

\subsection{Tooling}

We will use the tooling shown in 703, namely: Eclipse Epsilon~\cite{Epsilon} and
the languages it contains, and Xtext~\cite{Xtext}.

\newpage{}

%------------------------------------------------------------------------------
% Modelling
%------------------------------------------------------------------------------
\section{Modelling}
\label{sec:modelling}

\newpage{}

%------------------------------------------------------------------------------
% IDE
%------------------------------------------------------------------------------
\section{Integrated Development Environment}
\label{sec:integrated-development-environment}

\newpage{}

%------------------------------------------------------------------------------
% Model Validation
%------------------------------------------------------------------------------
\section{Model Validation}
\label{sec:model-validation}

\newpage{}

%------------------------------------------------------------------------------
% Model Management Operations
%------------------------------------------------------------------------------
\section{Model Management Operations}
\label{sec:model-management-operations}

\newpage{}

%------------------------------------------------------------------------------
% Reflection and Concluding Thoughts
%------------------------------------------------------------------------------
\section{Reflection and Concluding Thoughts}
\label{sec:reflection-and-concluding-thoughts}

\newpage{}

%------------------------------------------------------------------------------
% Bibliography
%------------------------------------------------------------------------------
\printbibliography[heading=bibintoc]

\end{document}
